\hypertarget{_command_line_CommandLineRunningTests}{}\section{\-Running Tests}\label{_command_line_CommandLineRunningTests}
\-To run the tests from the command line\-:


\begin{DoxyItemize}
\item \-Copy the \-Run\-Tests.\-sh (\href{http://github.com/gabriel/gh-unit/tree/master/Scripts/RunTests.sh}{\tt http\-://github.\-com/gabriel/gh-\/unit/tree/master/\-Scripts/\-Run\-Tests.\-sh}) file into your project directory (if you haven't already).
\item \-In \-X\-Code\-:
\begin{DoxyItemize}
\item \-To the {\ttfamily \-Tests} target, \-Add {\ttfamily \-New \-Build \-Phase} $|$ {\ttfamily \-New \-Run \-Script \-Build \-Phase}
\item \-Enter {\ttfamily sh \-Run\-Tests.\-sh} as the script. \-The path to {\ttfamily \-Run\-Tests.\-sh} should be relative to the xcode project file (.xcodeproj)!
\begin{DoxyItemize}
\item (\-Optional) \-Uncheck '\-Show environment variables in build log'
\end{DoxyItemize}
\end{DoxyItemize}
\end{DoxyItemize}

\-From the command line, run the tests from xcodebuild (with the \-G\-H\-U\-N\-I\-T\-\_\-\-C\-L\-I environment variable set)\-:

\begin{DoxyVerb}
 // For mac app; This might seg fault in 10.6, in which case you should use make test via Makefile below
 GHUNIT_CLI=1 xcodebuild -target Tests -configuration Debug -sdk macosx10.5 build	
 
 // For iPhone app
 GHUNIT_CLI=1 xcodebuild -target Tests -configuration Debug -sdk iphonesimulator4.0 build
 \end{DoxyVerb}


\-If you are wondering, the {\ttfamily \-Run\-Tests.\-sh} script will only run the tests if the env variable \-G\-H\-U\-N\-I\-T\-\_\-\-C\-L\-I is set. \-This is why this \-Run\-Script phase is ignored when running the test \-G\-U\-I. \-This is how we use a single \-Test target for both the \-G\-U\-I and command line testing.

\-This may seem strange that we run via xcodebuild with a \-Run\-Script phase in order to work on the command line, but otherwise we may not have the environment settings or other \-X\-Code specific configuration right.\hypertarget{_command_line_Makefile}{}\section{\-Makefile}\label{_command_line_Makefile}
\-Follow the directions above for adding command line support.

\-Example \-Makefile's for \-Mac or i\-Phone apps\-:


\begin{DoxyItemize}
\item \-Makefile (\-Mac \-O\-S \-X)\-: \href{http://github.com/gabriel/gh-unit/tree/master/Project/Makefile}{\tt http\-://github.\-com/gabriel/gh-\/unit/tree/master/\-Project/\-Makefile} (for a \-Mac \-App)
\item \-Makefile (i\-O\-S)\-: \href{http://github.com/gabriel/gh-unit/tree/master/Project-IPhone/Makefile}{\tt http\-://github.\-com/gabriel/gh-\/unit/tree/master/\-Project-\/\-I\-Phone/\-Makefile} (for an i\-Phone \-App)
\end{DoxyItemize}

\-The script will return a non-\/zero exit code on test failure.

\-To run the tests via the \-Makefile\-:

\begin{DoxyVerb}
 make test
 \end{DoxyVerb}
\hypertarget{_command_line_RunningATest}{}\section{\-Running a Test Case / Single Test}\label{_command_line_RunningATest}
\-The {\ttfamily \-T\-E\-S\-T} environment variable can be used to run a single test or test case.

\begin{DoxyVerb}
 // Run all tests in GHSlowTest
 make test TEST="GHSlowTest"
 
 // Run the method testSlowA in GHSlowTest	
 make test TEST="GHSlowTest/testSlowA"
 \end{DoxyVerb}
 