\-J\-S\-O\-N is mapped to \-Objective-\/\-C types in the following way\-:

\begin{DoxyItemize}
\item null -\/$>$ \-N\-S\-Null \item string -\/$>$ \hyperlink{class_n_s_string}{\-N\-S\-String} \item array -\/$>$ \-N\-S\-Mutable\-Array \item object -\/$>$ \-N\-S\-Mutable\-Dictionary \item true -\/$>$ \-N\-S\-Number's -\/number\-With\-Bool\-:\-Y\-E\-S \item false -\/$>$ \-N\-S\-Number's -\/number\-With\-Bool\-:\-N\-O \item integer up to 19 digits -\/$>$ \-N\-S\-Number's -\/number\-With\-Long\-Long\-: \item all other numbers -\/$>$ \-N\-S\-Decimal\-Number\end{DoxyItemize}
\-Since \-Objective-\/\-C doesn't have a dedicated class for boolean values, these turns into \-N\-S\-Number instances. \-However, since these are initialised with the -\/init\-With\-Bool\-: method they round-\/trip back to \-J\-S\-O\-N properly. \-In other words, they won't silently suddenly become 0 or 1; they'll be represented as 'true' and 'false' again.

\-As an optimisation integers up to 19 digits in length (the max length for signed long long integers) turn into \-N\-S\-Number instances, while complex ones turn into \-N\-S\-Decimal\-Number instances. \-We can thus avoid any loss of precision as \-J\-S\-O\-N allows ridiculously large numbers. 